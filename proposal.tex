\documentclass[pdftex,twocolumn,10pt,letterpaper]{article}
\usepackage{graphicx, times}
\usepackage{lipsum}

\setlength{\textheight}{9.0in}
\setlength{\columnsep}{0.25in}
\setlength{\textwidth}{6.50in}
\setlength{\topmargin}{0.0in}
\setlength{\headheight}{0.0in}
\setlength{\headsep}{0.0in}

\begin{document}
\title{Adaptable Scheduling Policy for Stream Processing System }
\author{
    Shawn Zhong, Suyan Qu, Sulong Zhou \\
    Group No. 7
}
\date{}

\interfootnotelinepenalty=10000

\maketitle

\section{Introduction}

The explosion of modern information and computer science technology systems has extremely accelerated the development of data science. A steady flow of high-volume data  

Existing scheduling policies proposed in Dhalion for stream processing systems focus on scaling up and down based on the current status of nodes, so actions can only be taken after symptoms are identified and diagnosed. However, in many cases, such need for scaling up and down can be predicted. For example, for a typical video streaming service, there would be more people watching at 7 pm when they finish work than at 4 am when they are still sleeping. Such daily workload is very predictable and we can scale up or down ahead of time to optimize resource usage and user experience. Our project aims to integrate policies to handle predictable traffic, with existing policies that handle sudden unprecedented change in traffic. This includes, but is not limited to, predicting future traffic from metric of each worker node, and taking actions based on predicted future traffic (either self-generated or user specified). 


Based on Dhalion~\cite{Floratou:2017:DSS:3137765.3137786}~\cite{Agrawal:2018:DAA:3229863.3275594}, we proposed ...

It integrates 


CloudLab~\cite{RicciEide}.
 
\section{Related Work}

Dhalion doesnot address 


Megaphone\cite{Hoffmann:2019:MLS:3329772.3342044} latency

Tutorial

\section{Timeline and Evaluation Plan}
For evaluating our project we plan to do the following:
\begin{itemize}
  \item Measure throughput, latency 
  \item Scale to 10 machines.
\end{itemize}

\begin{itemize}
  \item Nov 1: Run experiments 
  \item Dec 15: Write Final Report
\end{itemize}

{
\bibliographystyle{abbrv}
\bibliography{ref}
}
\end{document}
